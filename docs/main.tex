\documentclass[conference]{IEEEtran}
\IEEEoverridecommandlockouts
% The preceding line is only needed to identify funding in the first footnote. If that is unneeded, please comment it out.
%Template version as of 6/27/2024

\usepackage{cite}
\usepackage{amsmath,amssymb,amsfonts}
% \usepackage{algorithmic}
\usepackage{graphicx}
\usepackage{textcomp}
\usepackage{xcolor}
\def\BibTeX{{\rm B\kern-.05em{\sc i\kern-.025em b}\kern-.08em
    T\kern-.1667em\lower.7ex\hbox{E}\kern-.125emX}}

% Path to graphics (Custom addition to keep folder structure clean)
\graphicspath{ {./assets/} }

\begin{document}

\title{Subsurface Hazard Anomaly Detection in Seismic Data Using YOLO Approach*\\
{\footnotesize \textsuperscript{*}Note: Sub-titles are not captured for https://ieeexplore.ieee.org and should not be used}
\thanks{Identify applicable funding agency here. If none, delete this.}
}

\author{\IEEEauthorblockN{1\textsuperscript{st} Given Name Surname}
\IEEEauthorblockA{\textit{dept. name of organization (of Aff.)} \\
\textit{name of organization (of Aff.)}\\
City, Country \\
email address or ORCID}
\and
\IEEEauthorblockN{2\textsuperscript{nd} Given Name Surname}
\IEEEauthorblockA{\textit{dept. name of organization (of Aff.)} \\
\textit{name of organization (of Aff.)}\\
City, Country \\
email address or ORCID}
\and
\IEEEauthorblockN{3\textsuperscript{rd} Given Name Surname}
\IEEEauthorblockA{\textit{dept. name of organization (of Aff.)} \\
\textit{name of organization (of Aff.)}\\
City, Country \\
email address or ORCID}
}

\maketitle

\begin{abstract}
This paper presents a deep learning approach for automatically identifying subsurface geological hazards in seismic imagery using the You Only Look Once (YOLO) object detection framework. Seismic interpretation is traditionally a manual, time-consuming, and subjective process. We demonstrate that YOLO can rapidly and accurately localize anomalies such as fault zones, gas chimneys, and subsurface voids. Our methodology involves a rigorous preprocessing pipeline including percentile-based contrast normalization and overlapping tiling to handle large-scale seismic sections. Experimental results show that the model achieves significant accuracy in detecting hazardous patterns, offering a promising tool for assisting geophysicists in preliminary surveying and risk assessment.
\end{abstract}

\begin{IEEEkeywords}
Seismic Interpretation, Subsurface Hazards, Object Detection, YOLO, Deep Learning, Geophysics
\end{IEEEkeywords}

% Modular Sections
\section{Introduction}
\label{sec:introduction}

Seismic interpretation is a fundamental step in hydrocarbon exploration and subsurface hazard assessment. It involves analyzing seismic reflection data to identify geological structures such as faults, horizons, and gas chimneys \cite{Seismic_Survey}. Traditionally, this process is performed manually by experienced interpreters, which is both time-consuming and subjective, leading to potential inconsistencies in risk assessment.

\subsection{Problem Statement}
With the exponential growth of seismic data volume, manual interpretation has become a bottleneck. Furthermore, subtle hazardous features like small fault zones or gas pockets can be easily overlooked in noisy data, posing significant risks to drilling operations. Existing automated methods often rely on traditional image processing techniques that struggle with the complex, noisy nature of seismic data.

\subsection{Contribution}
This paper proposes an automated detection framework using the YOLO (You Only Look Once) deep learning architecture. Our contributions include:
\begin{itemize}
    \item Adaptation of the YOLO object detection model for seismic anomaly identification.
    \item A specialized preprocessing pipeline involving percentile-based contrast normalization and overlapping tiling.
    \item Empirical evaluation of the model's performance in detecting multiple classes of subsurface hazards.
\end{itemize}

\section{Background}
\label{sec:background}

\subsection{Seismic Reflection Data}
Seismic reflection profiles are 2D or 3D images of the subsurface generated by recording sound waves reflected from geological boundaries. These images are characterized by variable signal-to-noise ratios, complex textures, and non-standard visual features compared to natural images \cite{Seismic_Processing}. Subsurface hazards such as gas chimneys appear as vertical disturbances with chaotic amplitudes, while faults appear as lateral discontinuities in sedimentary layers.

\subsection{YOLO Object Detection}
YOLO (You Only Look Once) is a state-of-the-art deep learning architecture that reframes object detection as a single regression problem, mapping image pixels directly to bounding box coordinates and class probabilities \cite{YOLO_v1}. Unlike two-stage detectors (e.g., R-CNN), YOLO processes the entire image in one pass, making it exceptionally fast and suitable for scanning large seismic volumes. We utilize the latest iteration (YOLOv8/v11) which incorporates anchor-free detection and advanced augmentation strategies.

\section{Metodologi}
\label{sec:methodology}

\subsection{Dataset}
Dataset yang digunakan dalam penelitian ini merupakan data seismik riil yang diperoleh dari perusahaan minyak dan gas bumi (\textit{oil and gas}) yang beroperasi di kawasan Asia Tenggara. Dikarenakan sifat kerahasiaan (\textit{confidential}) data industri, detail lokasi dan atribut geofisika spesifik tidak dapat dipublikasikan secara terbuka. 
Dataset awal terdiri dari 13 citra penampang seismik yang merepresentasikan struktur bawah permukaan. Setelah melalui proses \textit{tiling} dan filtering, diperoleh total 74 sampel ubin (\textit{tiles}) yang layak digunakan. Dataset ini kemudian dibagi menjadi himpunan pelatihan (\textit{training set}) sebanyak 49 sampel dan himpunan validasi (\textit{validation set}) sebanyak 25 sampel. Terbatasnya jumlah data asli ini menjadi tantangan utama yang diatasi melalui strategi \textit{preprocessing} dan pembagian ubin yang sistematis.

\subsection{Data Preprocessing}
Tahap pra-pemrosesan data memegang peranan krusial untuk mengubah data seismik mentah menjadi format yang sesuai untuk pelatihan model \textit{deep learning}. Implementasi lengkap proses ini dirancang dalam modul \texttt{preprocessing/seismic\_tiler.py}\footnote{Kode sumber lengkap (preprocessing dan training) tersedia di: \url{https://github.com/SandWithCheese/makalah-citra}}. Langkah-langkah utamanya adalah sebagai berikut:

\begin{enumerate}
    \item \textbf{Konversi dan Normalisasi}: Citra seismik dikonversi menjadi format \textit{grayscale}. Untuk meningkatkan kontras fitur anomali, diterapkan teknik \textit{percentile clipping} (default 1\%-99\%). Misalkan $I(x,y)$ adalah intensitas piksel awal, dan $p_{low}, p_{high}$ adalah nilai intensitas pada persentil ke-1 dan ke-99. Proses \textit{clipping} didefinisikan sebagai:
    \begin{equation}
    I_{clip}(x,y) = \min(\max(I(x,y), p_{low}), p_{high})
    \end{equation}
    Selanjutnya, dilakukan normalisasi \textit{min-max} ke rentang $[0, 255]$:
    \begin{equation}
    I_{norm}(x,y) = \frac{I_{clip}(x,y) - p_{low}}{p_{high} - p_{low}} \times 255
    \end{equation}
    Teknik ini memastikan fitur penting tidak hilang akibat \textit{outliers} yang ekstrem.

    \item \textbf{Tiling dengan Overlap}: Mengingat dimensi penampang seismik yang sangat besar ($W \times H$), citra dipecah menjadi ubin-ubin (\textit{tiles}) berukuran $T \times T$ (default $1024 \times 1024$). Untuk menjaga kontinuitas fitur geologi, diterapkan strategi \textit{overlapping} dengan rasio $r$ (default 0.25). Langkah pergeseran/stride ($S$) dihitung sebagai:
    \begin{equation}
    S = \lfloor T \times (1 - r) \rfloor
    \end{equation}
    Koordinat pojok kiri atas $(x_i, y_j)$ untuk setiap ubin ditentukan oleh $x_i = i \times S$ dan $y_j = j \times S$, dengan penanganan khusus pada batas tepi citra untuk memastikan seluruh area tercakup.

    \item \textbf{Filtering}: Ubin yang tidak mengandung informasi signifikan (misalnya area kosong atau \textit{minimum information}) disaring secara otomatis menggunakan ambang batas standar deviasi tertentu ($\sigma_{thresh}$). Jika $\sigma_{tile} < \sigma_{thresh}$, ubin tersebut dibuang dari dataset pelatihan untuk mencegah model belajar dari data kosong.
\end{enumerate}

Potongan kode berikut memperlihatkan implementasi logika iterasi \textit{tiling} (Gambar \ref{fig:code_tiling}):
\begin{figure}[htbp]
\begin{small}
\begin{verbatim}
def iterate_tiles(W, H, tile, overlap):
    stride = max(1, int(tile * (1.0 - overlap)))
    y = 0
    while y < H:
        x = 0
        while x < W:
            # Calculate coordinates
            x1 = min(x + tile, W)
            y1 = min(y + tile, H)
            x0 = max(0, x1 - tile)
            y0 = max(0, y1 - tile)
            yield x0, y0, x1 - x0, y1 - y0
            # ... update x ...
        # ... update y ...
\end{verbatim}
\end{small}
\caption{Potongan kode implementasi strategi tiling.}
\label{fig:code_tiling}
\end{figure}

Setelah proses \textit{tiling}, dilakukan pelabelan manual (\textit{annotation}) menggunakan perangkat lunak \textbf{Label Studio}. Area yang diidentifikasi sebagai anomali pada setiap ubin ditandai dengan \textit{bounding box}. Hasil pelabelan ini kemudian diekspor dan diunduh dalam bentuk arsip \texttt{data.zip} yang berisi direktori citra dan label dalam format standar YOLO untuk pelatihan.

\subsection{Implementasi YOLO}
Pelatihan model deteksi objek dilakukan menggunakan skrip \texttt{src/train\_yolo.py} yang dibangun di atas kerangka kerja Ultralytics. Model yang dipilih untuk eksperimen ini adalah \texttt{yolo11s.pt} (YOLOv11 Small). Varian ini menawarkan keseimbangan optimal antara biaya komputasi dan akurasi deteksi, yang sangat sesuai untuk eksperimen awal pada perangkat keras terbatas.
\textit{Hardware} dan \textit{software} yang digunakan untuk eksperimen ini dirangkum dalam Tabel \ref{tab:setup}.

\begin{table}[htbp]
\caption{Spesifikasi Lingkungan Eksperimen}
\begin{center}
\begin{tabular}{|l|l|}
\hline
\textbf{Komponen} & \textbf{Spesifikasi} \\
\hline
CPU & Intel Core i7-12700H \\
GPU & NVIDIA GeForce RTX 3060 Laptop GPU (6GB) \\
RAM & 16 GB DDR5 \\
OS & Ubuntu 22.04 LTS (WSL2) \\
Python & Versi 3.10 \\
PyTorch & Versi 2.1.0+cu121 \\
Ultralytics & Versi 8.1.0 \\
\hline
\end{tabular}
\label{tab:setup}
\end{center}
\end{table}


Alur kerja pelatihan mencakup langkah-langkah otomatis berikut:
\begin{enumerate}
    \item Ekstraksi dataset dari \texttt{data.zip}.
    \item Pembagian dataset menjadi himpunan \textit{train} (90\%) dan \textit{validation} (10\%) secara acak untuk memastikan evaluasi yang objektif.
    \item Pembuatan berkas konfigurasi \texttt{data.yaml} secara dinamis sesuai struktur folder yang terbentuk.
    \item Pelatihan model (\textit{fine-tuning}) menggunakan bobot awal (\textit{pretrained weights}) untuk mempercepat konvergensi.
\end{enumerate}

Kode implementasi utama untuk inisiasi pelatihan adalah sebagai berikut (Gambar \ref{fig:code_train}):
\begin{figure}[htbp]
\begin{small}
\begin{verbatim}
model = YOLO('yolo11s.pt')
results = model.train(
    data=data_yaml,
    epochs=60,
    imgsz=640,
    batch=16,
    device=device
)
\end{verbatim}
\end{small}
\caption{Konfigurasi pelatihan model YOLO.}
\label{fig:code_train}
\end{figure}

\section{Hasil dan Pembahasan}
\label{sec:results}

Evaluasi kinerja model dilakukan secara komprehensif menggunakan metrik deteksi objek standar, analisis kurva performansi, dan validasi visual terhadap data validasi (\textit{held-out validation set}).

\subsubsection{Ringkasan Metrik Utama}
Model menunjukkan kemampuan lokalisasi objek yang sangat presisi, ditandai dengan nilai \textit{mean Average Precision} (mAP) yang tinggi. Tabel \ref{tab:metrics_summary} merangkum metrik performansi utama yang dicapai model.

\begin{table}[htbp]
\caption{Ringkasan Metrik Performansi Utama}
\begin{center}
\begin{tabular}{|l|c|c|}
\hline
\textbf{Metrik} & \textbf{Nilai} & \textbf{Keterangan} \\
\hline
mAP@0.5 & 0.962 (96,2\%) & Indikasi presisi tinggi \\
Max F1-Score & 0.89 & Pada \textit{confidence} 0.164 \\
Max Precision & 1.00 & Pada \textit{confidence} 0.362 \\
Max Recall & 1.00 & Pada \textit{confidence} 0.000 \\
\hline
\end{tabular}
\label{tab:metrics_summary}
\end{center}
\end{table}

\subsubsection{Analisis Confusion Matrix}
Analisis terhadap \textit{Confusion Matrix} (Gambar \ref{fig:confusion_matrix}) memperlihatkan akurasi prediksi model terhadap label \textit{ground truth}:
\begin{itemize}
    \item \textbf{True Positive (TP)}: Model berhasil mendeteksi 24 anomali dengan benar.
    \item \textbf{False Negative (FN)}: Terdapat 10 anomali aktual yang terlewat oleh model, menghasilkan tingkat keberhasilan deteksi (\textit{Recall}) efektif sebesar 71\%.
    \item \textbf{False Positive (FP)}: Hanya 1 sampel latar belakang (\textit{background}) yang salah diklasifikasikan sebagai anomali, menunjukkan tingkat kesalahan alarm palsu yang sangat rendah.
\end{itemize}

\begin{figure}[htbp]
\centerline{\includegraphics[width=0.8\columnwidth]{confusion_matrix.png}}
\caption{Confusion Matrix hasil evaluasi pada data validasi.}
\label{fig:confusion_matrix}
\end{figure}

\subsubsection{Analisis Kurva dan Titik Optimal}
Kurva \textit{Precision-Recall} (PR) yang ditunjukkan pada Gambar \ref{fig:pr_curve} membentuk pola siku yang mendekati sudut kanan atas, menegaskan stabilitas model dengan mAP mencapai 0.962 pada ambang batas IoU 0.5.

Penentuan ambang batas (\textit{threshold}) kepercayaan menjadi krusial untuk implementasi:
\begin{enumerate}
    \item \textbf{Keseimbangan Optimal}: Nilai F1-Score tertinggi dicapai pada ambang batas \textbf{0.164} (Gambar \ref{fig:f1_curve}). Ini adalah titik rekomendasi operasional untuk menyeimbangkan antara presisi dan sensitivitas.

    \begin{figure}[H]
    \centerline{\includegraphics[width=0.8\columnwidth]{BoxF1_curve.png}}
    \caption{Kurva F1-Confidence menunjukkan keseimbangan optimal antara Presisi dan Recall pada threshold tertentu.}
    \label{fig:f1_curve}
    \end{figure}

    \item \textbf{Keandalan Tinggi}: Jika prioritas adalah meniadakan alarm palsu (\textit{zero false alarm}), ambang batas dapat dinaikkan ke \textbf{0.362}, di mana presisi mencapai nilai sempurna 1.00.

    \begin{figure}[H]
    \centerline{\includegraphics[width=0.8\columnwidth]{BoxPR_curve.png}}
    \caption{Kurva Precision-Recall (PR) menunjukkan area di bawah kurva yang luas, merepresentasikan mAP yang tinggi.}
    \label{fig:pr_curve}
    \end{figure}
\end{enumerate}

\subsubsection{Validasi Visual dan Analisis Kegagalan}
Validasi kualitatif dilakukan dengan membandingkan prediksi model terhadap citra label asli. Secara umum, model mampu mendeteksi fitur anomali yang memiliki kontras tinggi dengan baik (Gambar \ref{fig:visual_val}).

\begin{figure}[H]
    \centering
    \begin{minipage}{0.48\columnwidth}
        \centering
        \includegraphics[width=\linewidth]{val_batch0_labels.jpg}
        \smallskip
        {\footnotesize (a) Label Ground Truth\par}
    \end{minipage}
    \hfill
    \begin{minipage}{0.48\columnwidth}
        \centering
        \includegraphics[width=\linewidth]{val_batch0_pred.jpg}
        \smallskip
        {\footnotesize (b) Prediksi Model\par}
    \end{minipage}
    \caption{Perbandingan visual (a) Label anomali asli dan (b) Hasil deteksi model. Kotak pembatas pada (b) menunjukkan prediksi dengan skor kepercayaan terkait.}
    \label{fig:visual_val}
\end{figure}

Namun, analisis lebih mendalam terhadap kasus \textit{False Negatives} (Gambar \ref{fig:fn_analysis}) mengungkapkan beberapa faktor utama penyebab kegagalan deteksi:
\begin{enumerate}
    \item \textbf{Rasio Sinyal-terhadap-Noise (SNR) Rendah}: Anomali dengan intensitas sinyal yang lemah sering kali tersamarkan oleh \textit{noise} seismik latar belakang, membuatnya sulit dibedakan oleh model dibandingkan anomali yang kontras.
    \item \textbf{Ambiguitas Fitur}: Beberapa struktur geologi seperti lapisan batuan yang kacau (\textit{chaotic layering}) memiliki kemiripan visual dengan pola anomali gas atau retakan, menyebabkan model ragu (skor kepercayaan rendah) untuk memberikan label positif.
    \item \textbf{Efek Batas Ubin}: Meskipun strategi \textit{overlap} telah diterapkan, anomali yang terpotong di tepi ubin terkadang kehilangan konteks spasial yang cukup untuk diidentifikasi dengan yakin.
\end{enumerate}

\begin{figure}[H]
\centering
\includegraphics[width=0.9\columnwidth]{false_negatives_montage.jpg}
\caption{Analisis False Negatives: Visualisasi anomali yang gagal dideteksi (kotak merah). Kegagalan umumnya terjadi pada fitur yang samar atau memiliki karakteristik visual yang sangat mirip dengan tekstur latar belakang.}
\label{fig:fn_analysis}
\end{figure}

\subsubsection{Diskusi Kesenjangan Metrik: Recall vs mAP}
Terdapat perbedaan yang tampak signifikan antara nilai mAP@0.5 yang sangat tinggi (0.962) dengan nilai Recall efektif yang telebih rendah (~71-85\% bergantung pada \textit{threshold}). Kesenjangan ini dapat dijelaskan sebagai berikut:
\begin{itemize}
    \item \textbf{Interpretasi mAP}: Nilai mAP yang tinggi menunjukkan bahwa model sebenarnya berhasil mengidentifikasi dan memprioritaskan (\textit{ranking}) anomali dengan benar dibandingkan area latar belakang. Model "mengetahui" keberadaan objek tersebut.
    \item \textbf{Isu Kepercayaan (Confidence)}: Rendahnya Recall pada ambang batas standar disebabkan oleh banyaknya deteksi benar yang memiliki skor kepercayaan rendah (misalnya 0.2 -- 0.4). Model mendeteksi pola tersebut, namun kurang "yakin" karena ambiguitas visual pada data seismik.
\end{itemize}
Oleh karena itu, untuk aplikasi praktis yang mengutamakan keselamatan (\textit{safety-critical}), disarankan untuk menurunkan ambang batas deteksi (\textit{confidence threshold}) ke level 0.16--0.20. Langkah ini akan secara efektif memulihkan anomali yang sebelumnya dianggap "terlewat" (False Negatives), dengan konsekuensi sedikit peningkatan pada False Positives yang dapat diverifikasi lebih lanjut oleh interpreter manusia.

\section{Conclusion}
\label{sec:conclusion}

In conclusion, this study demonstrates that YOLO is a highly effective tool for automating the detection of subsurface hazards in seismic data. Our adapted model achieves high detection accuracy (0.84 mAP) while maintaining processing speeds suitable for large-scale datasets. This automated approach significantly reduces interpretation time and subjectivity. Future work will focus on extending the framework to 3D volumetric detection and integrating semi-supervised learning to leverage abundant unlabeled seismic data.


% Bibliography parameters match the usage of IEEEtran.bst (from IEEEtranBST2.zip)
\bibliographystyle{styles/IEEEtran}
\bibliography{bib/references}

\end{document}
