\subsection{Evaluasi Model}
\label{sec:results}

Model dievaluasi menggunakan \textit{held-out test set} dari ubin seismik yang tidak dilibatkan dalam pelatihan. Kami melaporkan metrik \textit{object detection} standar: \textit{Precision} (P), \textit{Recall} (R), dan \textit{mean Average Precision} (mAP) pada ambang batas \textit{Intersection over Union} (IoU) sebesar 0.5.

\begin{figure}[htbp]
\centerline{\includegraphics[width=\columnwidth]{detections.jpg}}
\caption{Visualisasi hasil prediksi model pada data validasi. Kotak pembatas berwarna menunjukkan lokasi anomali seismik yang terdeteksi (\textit{Fault}, \textit{Gas Chimney}, \textit{Void}) dibandingkan dengan label \textit{ground truth}.}
\label{fig:detections}
\end{figure}

\begin{table}[htbp]
\caption{Kinerja Deteksi Berdasarkan Kelas}
\begin{center}
\begin{tabular}{|c|c|c|c|}
\hline
\textbf{Kelas} & \textbf{\textit{Precision}} & \textbf{\textit{Recall}} & \textbf{mAP@0.5} \\
\hline
\textit{Fault} & 0.85 & 0.78 & 0.82 \\
\textit{Gas Chimney} & 0.76 & 0.82 & 0.79 \\
\textit{Void} & 0.91 & 0.88 & 0.90 \\
\hline
\textbf{Keseluruhan} & \textbf{0.84} & \textbf{0.83} & \textbf{0.84} \\
\hline
\end{tabular}
\label{tab:metrics}
\end{center}
\end{table}

Analisis kualitatif, seperti yang ditunjukkan pada Gambar \ref{fig:detections}, menegaskan kemampuan model untuk melokalisasi fitur bahaya secara visual. \textit{Bounding box} yang dihasilkan secara akurat mencakup jangkauan vertikal \textit{gas chimneys} dan diskontinuitas linear \textit{fault zones}, membedakannya secara efektif dari lapisan sedimen di sekitarnya meskipun terdapat derau pada data.
