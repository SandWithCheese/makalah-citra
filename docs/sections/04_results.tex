\section{Results and Analysis}
\label{sec:results}

\subsection{Quantitative Analysis}
The model was evaluated on a held-out test set of seismic tiles. We report standard object detection metrics: Precision (P), Recall (R), and mean Average Precision (mAP) at an Intersection over Union (IoU) threshold of 0.5.
\begin{table}[htbp]
\caption{Detection Performance by Class}
\begin{center}
\begin{tabular}{|c|c|c|c|}
\hline
\textbf{Class} & \textbf{Precision} & \textbf{Recall} & \textbf{mAP@0.5} \\
\hline
Fault & 0.85 & 0.78 & 0.82 \\
Gas Chimney & 0.76 & 0.82 & 0.79 \\
Void & 0.91 & 0.88 & 0.90 \\
\hline
\textbf{Overall} & \textbf{0.84} & \textbf{0.83} & \textbf{0.84} \\
\hline
\end{tabular}
\label{tab:metrics}
\end{center}
\end{table}

\subsection{Qualitative Analysis}
Figure \ref{fig:detections} demonstrates the model's ability to localize hazard features. The bounding boxes accurately encompass the vertical extent of gas chimneys and the linear discontinuities of fault zones, distinguishing them from surrounding sedimentary layers.
