\section{Hasil dan Pembahasan}
\label{sec:results}

Evaluasi kinerja model dilakukan secara komprehensif menggunakan metrik deteksi objek standar, analisis kurva performansi, dan validasi visual terhadap data validasi (\textit{held-out validation set}).

\subsubsection{Ringkasan Metrik Utama}
Model menunjukkan kemampuan lokalisasi objek yang sangat presisi, ditandai dengan nilai \textit{mean Average Precision} (mAP) yang tinggi. Tabel \ref{tab:metrics_summary} merangkum metrik performansi utama yang dicapai model.

\begin{table}[htbp]
\caption{Ringkasan Metrik Performansi Utama}
\begin{center}
\begin{tabular}{|l|c|c|}
\hline
\textbf{Metrik} & \textbf{Nilai} & \textbf{Keterangan} \\
\hline
mAP@0.5 & 0.962 (96,2\%) & Indikasi presisi tinggi \\
Max F1-Score & 0.89 & Pada \textit{confidence} 0.164 \\
Max Precision & 1.00 & Pada \textit{confidence} 0.362 \\
Max Recall & 1.00 & Pada \textit{confidence} 0.000 \\
\hline
\end{tabular}
\label{tab:metrics_summary}
\end{center}
\end{table}

\subsubsection{Analisis Confusion Matrix}
Analisis terhadap \textit{Confusion Matrix} (Gambar \ref{fig:confusion_matrix}) memperlihatkan akurasi prediksi model terhadap label \textit{ground truth}:
\begin{itemize}
    \item \textbf{True Positive (TP)}: Model berhasil mendeteksi 24 anomali dengan benar.
    \item \textbf{False Negative (FN)}: Terdapat 10 anomali aktual yang terlewat oleh model, menghasilkan tingkat keberhasilan deteksi (\textit{Recall}) efektif sebesar 71\%.
    \item \textbf{False Positive (FP)}: Hanya 1 sampel latar belakang (\textit{background}) yang salah diklasifikasikan sebagai anomali, menunjukkan tingkat kesalahan alarm palsu yang sangat rendah.
\end{itemize}

\begin{figure}[htbp]
\centerline{\includegraphics[width=0.8\columnwidth]{confusion_matrix.png}}
\caption{Confusion Matrix hasil evaluasi pada data validasi.}
\label{fig:confusion_matrix}
\end{figure}

\subsubsection{Analisis Kurva dan Titik Optimal}
Kurva \textit{Precision-Recall} (PR) yang ditunjukkan pada Gambar \ref{fig:pr_curve} membentuk pola siku yang mendekati sudut kanan atas, menegaskan stabilitas model dengan mAP mencapai 0.962 pada ambang batas IoU 0.5.

Penentuan ambang batas (\textit{threshold}) kepercayaan menjadi krusial untuk implementasi:
\begin{enumerate}
    \item \textbf{Keseimbangan Optimal}: Nilai F1-Score tertinggi dicapai pada ambang batas \textbf{0.164}. Ini adalah titik rekomendasi operasional untuk menyeimbangkan antara presisi dan sensitivitas.

    \begin{figure}[H]
    \centerline{\includegraphics[width=0.8\columnwidth]{BoxF1_curve.png}}
    \caption{Kurva F1-Confidence menunjukkan keseimbangan optimal antara Presisi dan Recall pada threshold tertentu.}
    \label{fig:f1_curve}
    \end{figure}

    \item \textbf{Keandalan Tinggi}: Jika prioritas adalah meniadakan alarm palsu (\textit{zero false alarm}), ambang batas dapat dinaikkan ke \textbf{0.362}, di mana presisi mencapai nilai sempurna 1.00.

    \begin{figure}[H]
    \centerline{\includegraphics[width=0.8\columnwidth]{BoxPR_curve.png}}
    \caption{Kurva Precision-Recall (PR) menunjukkan area di bawah kurva yang luas, merepresentasikan mAP yang tinggi.}
    \label{fig:pr_curve}
    \end{figure}
\end{enumerate}

\subsubsection{Validasi Visual}
Validasi kualitatif dilakukan dengan membandingkan prediksi model terhadap citra label asli (Gambar \ref{fig:visual_val}). Observasi pada sampel validasi (\textit{batch} 0 dan 1) menunjukkan bahwa:
\begin{itemize}
    \item Model sangat handal dalam mendeteksi fitur anomali yang kontras dan jelas.
    \item Anomali yang gagal terdeteksi (\textit{False Negative}) umumnya memiliki skor kepercayaan (\textit{confidence score}) yang rendah (0.3 - 0.4), sehingga tertolak jika ambang batas deteksi diset terlalu tinggi.

    \begin{figure}[H]
    \centering
    \begin{minipage}{0.48\columnwidth}
        \centering
        \includegraphics[width=\linewidth]{val_batch0_labels.jpg}
        \smallskip
        {\footnotesize (a) Label Ground Truth\par}
    \end{minipage}
    \hfill
    \begin{minipage}{0.48\columnwidth}
        \centering
        \includegraphics[width=\linewidth]{val_batch0_pred.jpg}
        \smallskip
        {\footnotesize (b) Prediksi Model\par}
    \end{minipage}
    \caption{Perbandingan visual (a) Label anomali asli dan (b) Hasil deteksi model. Kotak pembatas pada (b) menunjukkan prediksi dengan skor kepercayaan terkait.}
    \label{fig:visual_val}
    \end{figure}
\end{itemize}

Secara keseluruhan, model direkomendasikan untuk digunakan dengan \textit{confidence threshold} 0.16 untuk meminimalkan anomali yang terlewat, sembari tetap menjaga tingkat \textit{false positive} yang minimal.
