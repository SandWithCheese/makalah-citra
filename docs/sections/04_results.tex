\section{Hasil dan Pembahasan}
\label{sec:results}

Evaluasi kinerja model dilakukan secara komprehensif menggunakan metrik deteksi objek standar, analisis kurva performansi, dan validasi visual terhadap data validasi (\textit{held-out validation set}).

\subsubsection{Ringkasan Metrik Utama}
Model menunjukkan kemampuan lokalisasi objek yang sangat presisi, ditandai dengan nilai \textit{mean Average Precision} (mAP) yang tinggi. Tabel \ref{tab:metrics_summary} merangkum metrik performansi utama yang dicapai model.

\begin{table}[htbp]
\caption{Ringkasan Metrik Performansi Utama}
\begin{center}
\begin{tabular}{|l|c|c|}
\hline
\textbf{Metrik} & \textbf{Nilai} & \textbf{Keterangan} \\
\hline
mAP@0.5 & 0.962 (96,2\%) & Indikasi presisi tinggi \\
Max F1-Score & 0.89 & Pada \textit{confidence} 0.164 \\
Max Precision & 1.00 & Pada \textit{confidence} 0.362 \\
Max Recall & 1.00 & Pada \textit{confidence} 0.000 \\
\hline
\end{tabular}
\label{tab:metrics_summary}
\end{center}
\end{table}

\subsubsection{Analisis Confusion Matrix}
Analisis terhadap \textit{Confusion Matrix} (Gambar \ref{fig:confusion_matrix}) memperlihatkan akurasi prediksi model terhadap label \textit{ground truth}:
\begin{itemize}
    \item \textbf{True Positive (TP)}: Model berhasil mendeteksi 24 anomali dengan benar.
    \item \textbf{False Negative (FN)}: Terdapat 10 anomali aktual yang terlewat oleh model, menghasilkan tingkat keberhasilan deteksi (\textit{Recall}) efektif sebesar 71\%.
    \item \textbf{False Positive (FP)}: Hanya 1 sampel latar belakang (\textit{background}) yang salah diklasifikasikan sebagai anomali, menunjukkan tingkat kesalahan alarm palsu yang sangat rendah.
\end{itemize}

\begin{figure}[htbp]
\centerline{\includegraphics[width=0.8\columnwidth]{confusion_matrix.png}}
\caption{Confusion Matrix hasil evaluasi pada data validasi.}
\label{fig:confusion_matrix}
\end{figure}

\subsubsection{Analisis Kurva dan Titik Optimal}
Kurva \textit{Precision-Recall} (PR) yang ditunjukkan pada Gambar \ref{fig:pr_curve} membentuk pola siku yang mendekati sudut kanan atas, menegaskan stabilitas model dengan mAP mencapai 0.962 pada ambang batas IoU 0.5.

Penentuan ambang batas (\textit{threshold}) kepercayaan menjadi krusial untuk implementasi:
\begin{enumerate}
    \item \textbf{Keseimbangan Optimal}: Nilai F1-Score tertinggi dicapai pada ambang batas \textbf{0.164} (Gambar \ref{fig:f1_curve}). Ini adalah titik rekomendasi operasional untuk menyeimbangkan antara presisi dan sensitivitas.

    \begin{figure}[H]
    \centerline{\includegraphics[width=0.8\columnwidth]{BoxF1_curve.png}}
    \caption{Kurva F1-Confidence menunjukkan keseimbangan optimal antara Presisi dan Recall pada threshold tertentu.}
    \label{fig:f1_curve}
    \end{figure}

    \item \textbf{Keandalan Tinggi}: Jika prioritas adalah meniadakan alarm palsu (\textit{zero false alarm}), ambang batas dapat dinaikkan ke \textbf{0.362}, di mana presisi mencapai nilai sempurna 1.00.

    \begin{figure}[H]
    \centerline{\includegraphics[width=0.8\columnwidth]{BoxPR_curve.png}}
    \caption{Kurva Precision-Recall (PR) menunjukkan area di bawah kurva yang luas, merepresentasikan mAP yang tinggi.}
    \label{fig:pr_curve}
    \end{figure}
\end{enumerate}

\subsubsection{Validasi Visual dan Analisis Kegagalan}
Validasi kualitatif dilakukan dengan membandingkan prediksi model terhadap citra label asli. Secara umum, model mampu mendeteksi fitur anomali yang memiliki kontras tinggi dengan baik (Gambar \ref{fig:visual_val}).

\begin{figure}[H]
    \centering
    \begin{minipage}{0.48\columnwidth}
        \centering
        \includegraphics[width=\linewidth]{val_batch0_labels.jpg}
        \smallskip
        {\footnotesize (a) Label Ground Truth\par}
    \end{minipage}
    \hfill
    \begin{minipage}{0.48\columnwidth}
        \centering
        \includegraphics[width=\linewidth]{val_batch0_pred.jpg}
        \smallskip
        {\footnotesize (b) Prediksi Model\par}
    \end{minipage}
    \caption{Perbandingan visual (a) Label anomali asli dan (b) Hasil deteksi model. Kotak pembatas pada (b) menunjukkan prediksi dengan skor kepercayaan terkait.}
    \label{fig:visual_val}
\end{figure}

Namun, analisis lebih mendalam terhadap kasus \textit{False Negatives} (Gambar \ref{fig:fn_analysis}) mengungkapkan beberapa faktor utama penyebab kegagalan deteksi:
\begin{enumerate}
    \item \textbf{Rasio Sinyal-terhadap-Noise (SNR) Rendah}: Anomali dengan intensitas sinyal yang lemah sering kali tersamarkan oleh \textit{noise} seismik latar belakang, membuatnya sulit dibedakan oleh model dibandingkan anomali yang kontras.
    \item \textbf{Ambiguitas Fitur}: Beberapa struktur geologi seperti lapisan batuan yang kacau (\textit{chaotic layering}) memiliki kemiripan visual dengan pola anomali gas atau retakan, menyebabkan model ragu (skor kepercayaan rendah) untuk memberikan label positif.
    \item \textbf{Efek Batas Ubin}: Meskipun strategi \textit{overlap} telah diterapkan, anomali yang terpotong di tepi ubin terkadang kehilangan konteks spasial yang cukup untuk diidentifikasi dengan yakin.
\end{enumerate}

\begin{figure}[H]
\centering
\includegraphics[width=0.9\columnwidth]{false_negatives_montage.jpg}
\caption{Analisis False Negatives: Visualisasi anomali yang gagal dideteksi (kotak merah). Kegagalan umumnya terjadi pada fitur yang samar atau memiliki karakteristik visual yang sangat mirip dengan tekstur latar belakang.}
\label{fig:fn_analysis}
\end{figure}

\subsubsection{Diskusi Kesenjangan Metrik: Recall vs mAP}
Terdapat perbedaan yang tampak signifikan antara nilai mAP@0.5 yang sangat tinggi (0.962) dengan nilai Recall efektif yang telebih rendah (~71-85\% bergantung pada \textit{threshold}). Kesenjangan ini dapat dijelaskan sebagai berikut:
\begin{itemize}
    \item \textbf{Interpretasi mAP}: Nilai mAP yang tinggi menunjukkan bahwa model sebenarnya berhasil mengidentifikasi dan memprioritaskan (\textit{ranking}) anomali dengan benar dibandingkan area latar belakang. Model "mengetahui" keberadaan objek tersebut.
    \item \textbf{Isu Kepercayaan (Confidence)}: Rendahnya Recall pada ambang batas standar disebabkan oleh banyaknya deteksi benar yang memiliki skor kepercayaan rendah (misalnya 0.2 -- 0.4). Model mendeteksi pola tersebut, namun kurang "yakin" karena ambiguitas visual pada data seismik.
\end{itemize}
Oleh karena itu, untuk aplikasi praktis yang mengutamakan keselamatan (\textit{safety-critical}), disarankan untuk menurunkan ambang batas deteksi (\textit{confidence threshold}) ke level 0.16--0.20. Langkah ini akan secara efektif memulihkan anomali yang sebelumnya dianggap "terlewat" (False Negatives), dengan konsekuensi sedikit peningkatan pada False Positives yang dapat diverifikasi lebih lanjut oleh interpreter manusia.
