\section{Metodologi}
\label{sec:methodology}

\subsection{Dataset}
Dataset yang digunakan dalam penelitian ini merupakan data seismik riil yang diperoleh dari perusahaan minyak dan gas bumi (\textit{oil and gas}) yang beroperasi di kawasan Asia Tenggara. Dikarenakan sifat kerahasiaan (\textit{confidential}) data industri, detail lokasi dan atribut geofisika spesifik tidak dapat dipublikasikan secara terbuka. 
Dataset awal terdiri dari 13 citra penampang seismik yang merepresentasikan struktur bawah permukaan. Setelah melalui proses \textit{tiling} dan filtering, diperoleh total 74 sampel ubin (\textit{tiles}) yang layak digunakan. Dataset ini kemudian dibagi menjadi himpunan pelatihan (\textit{training set}) sebanyak 49 sampel dan himpunan validasi (\textit{validation set}) sebanyak 25 sampel. Terbatasnya jumlah data asli ini menjadi tantangan utama yang diatasi melalui strategi \textit{preprocessing} dan pembagian ubin yang sistematis.

\subsection{Data Preprocessing}
Tahap pra-pemrosesan data memegang peranan krusial untuk mengubah data seismik mentah menjadi format yang sesuai untuk pelatihan model \textit{deep learning}. Implementasi lengkap proses ini dirancang dalam modul \texttt{preprocessing/seismic\_tiler.py}. Langkah-langkah utamanya adalah sebagai berikut:

\begin{enumerate}
    \item \textbf{Konversi dan Normalisasi}: Citra seismik dikonversi menjadi format \textit{grayscale}. Untuk meningkatkan kontras fitur anomali, diterapkan teknik \textit{percentile clipping} (default 1\%-99\%), di mana intensitas piksel di luar rentang ini dipotong (\textit{clipped}). Hasilnya kemudian dinormalisasi ke rentang intensitas [0, 255] agar sesuai dengan format citra 8-bit.
    \item \textbf{Tiling dengan Overlap}: Mengingat dimensi penampang seismik yang sangat besar, citra dipecah menjadi ubin-ubin (\textit{tiles}) berukuran lebih kecil, dengan ukuran default $1024 \times 1024$ piksel. Untuk menjaga kontinuitas fitur geologi yang mungkin terpotong di batas ubin, diterapkan strategi \textit{overlapping} sebesar 25\% antar ubin.
    \item \textbf{Filtering}: Ubin yang tidak mengandung informasi signifikan (misalnya area kosong atau \textit{minimum information}) disaring secara otomatis menggunakan ambang batas standar deviasi tertentu.
\end{enumerate}

Potongan kode berikut memperlihatkan implementasi logika iterasi \textit{tiling} (Gambar \ref{fig:code_tiling}):
\begin{figure}[htbp]
\begin{small}
\begin{verbatim}
def iterate_tiles(W, H, tile, overlap):
    stride = max(1, int(tile * (1.0 - overlap)))
    y = 0
    while y < H:
        x = 0
        while x < W:
            # Calculate coordinates
            x1 = min(x + tile, W)
            y1 = min(y + tile, H)
            x0 = max(0, x1 - tile)
            y0 = max(0, y1 - tile)
            yield x0, y0, x1 - x0, y1 - y0
            # ... update x ...
        # ... update y ...
\end{verbatim}
\end{small}
\caption{Potongan kode implementasi strategi tiling.}
\label{fig:code_tiling}
\end{figure}

Setelah proses \textit{tiling}, dilakukan pelabelan manual (\textit{annotation}) menggunakan perangkat lunak \textbf{Label Studio}. Area yang diidentifikasi sebagai anomali pada setiap ubin ditandai dengan \textit{bounding box}. Hasil pelabelan ini kemudian diekspor dan diunduh dalam bentuk arsip \texttt{data.zip} yang berisi direktori citra dan label dalam format standar YOLO untuk pelatihan.

\subsection{Implementasi YOLO}
Pelatihan model deteksi objek dilakukan menggunakan skrip \texttt{src/train\_yolo.py} yang dibangun di atas kerangka kerja Ultralytics. Model yang dipilih untuk eksperimen ini adalah \texttt{yolo11s.pt} (YOLOv11 Small). Varian ini menawarkan keseimbangan optimal antara biaya komputasi dan akurasi deteksi, yang sangat sesuai untuk eksperimen awal pada perangkat keras terbatas.

Alur kerja pelatihan mencakup langkah-langkah otomatis berikut:
\begin{enumerate}
    \item Ekstraksi dataset dari \texttt{data.zip}.
    \item Pembagian dataset menjadi himpunan \textit{train} (90\%) dan \textit{validation} (10\%) secara acak untuk memastikan evaluasi yang objektif.
    \item Pembuatan berkas konfigurasi \texttt{data.yaml} secara dinamis sesuai struktur folder yang terbentuk.
    \item Pelatihan model (\textit{fine-tuning}) menggunakan bobot awal (\textit{pretrained weights}) untuk mempercepat konvergensi.
\end{enumerate}

Kode implementasi utama untuk inisiasi pelatihan adalah sebagai berikut (Gambar \ref{fig:code_train}):
\begin{figure}[htbp]
\begin{small}
\begin{verbatim}
model = YOLO('yolo11s.pt')
results = model.train(
    data=data_yaml,
    epochs=60,
    imgsz=640,
    batch=16,
    device=device
)
\end{verbatim}
\end{small}
\caption{Konfigurasi pelatihan model YOLO.}
\label{fig:code_train}
\end{figure}
