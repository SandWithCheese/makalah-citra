\section{Kesimpulan}
\label{sec:conclusion}

Penelitian ini menunjukkan bahwa YOLO merupakan alat yang efektif untuk mengotomatisasi deteksi \textit{subsurface hazards} pada data seismik. Model yang kami adaptasi mencapai akurasi deteksi yang signifikan (0.84 mAP) dengan tetap mempertahankan kecepatan pemrosesan yang sesuai untuk dataset berskala besar. Pendekatan otomatis ini berpotensi mengurangi waktu interpretasi dan subjektivitas secara signifikan. Pekerjaan di masa depan akan berfokus pada perluasan kerangka kerja ke \textit{3D volumetric detection} dan pengintegrasian \textit{semi-supervised learning} untuk memanfaatkan data seismik tak berlabel yang melimpah.
