\section{Kesimpulan}
\label{sec:conclusion}

Berdasarkan hasil eksperimen dan analisis yang telah dilakukan, dapat disimpulkan bahwa implementasi metode \textit{You Only Look Once} (YOLO) efektif dalam mendeteksi anomali bahaya bawah permukaan (\textit{subsurface hazards}) pada citra seismik. Model berhasil mencapai kinerja deteksi yang sangat presisi dengan nilai \textit{mean Average Precision} (mAP@0.5) sebesar 0.962. Evaluasi pada \textit{confusion matrix} menunjukkan kemampuan model yang kuat dalam meminimalisir kesalahan deteksi palsu (\textit{false positives}), dengan hanya ditemukan satu kesalahan identifikasi latar belakang sebagai anomali pada data validasi.

Meskipun presisi yang dihasilkan sangat tinggi, model masih memiliki keterbatasan dalam mengenali anomali dengan kontras rendah atau fitur visual yang samar, sebagaimana ditunjukkan oleh adanya sejumlah kasus \textit{False Negative}. Berdasarkan analisis kurva F1, pengaturan ambang batas kepercayaan (\textit{confidence threshold}) pada nilai 0.164 direkomendasikan sebagai titik operasional optimal. Pada titik ini, model mencapai keseimbangan terbaik antara sensitivitas dan presisi dengan skor F1 maksimal sebesar 0.89, yang cukup memadai untuk fungsi penapisan awal dalam alur kerja interpretasi seismik.

Untuk pengembangan penelitian selanjutnya, disarankan untuk memperluas dataset pelatihan dengan variasi augmentasi kontras yang lebih agresif guna meningkatkan sensitivitas model terhadap anomali yang samar. Selain itu, transisi menuju pendekatan deteksi volumetrik 3D perlu dipertimbangkan untuk memanfaatkan informasi kontinuitas spasial antar-slice yang tidak tertangkap pada analisis 2D. Eksplorasi teknik \textit{semi-supervised learning} juga berpotensi meningkatkan generalisasi model dengan memanfaatkan ketersediaan data seismik tak berlabel yang melimpah. Penelitian ini menegaskan potensi besar pendekatan berbasis visi komputer sebagai alat bantu interpretasi geofisika, namun hasil deteksi tetap memerlukan validasi ahli untuk pengambilan keputusan eksplorasi.
