\section{Pendahuluan}
\label{sec:introduction}

\textit{Seismic interpretation} merupakan langkah fundamental dalam eksplorasi hidrokarbon dan penilaian \textit{subsurface hazard}. Proses ini melibatkan analisis data refleksi seismik untuk mengidentifikasi struktur geologi seperti \textit{faults}, \textit{horizons}, dan \textit{gas chimneys} \cite{RemoteSensing2023}. Secara tradisional, proses ini dilakukan secara manual oleh interpreter berpengalaman, yang memakan waktu dan bersifat subjektif, sehingga berpotensi menimbulkan inkonsistensi dalam penilaian risiko.

Seiring dengan pertumbuhan volume data seismik yang eksponensial, interpretasi manual menjadi hambatan utama (\textit{bottleneck}). Selain itu, fitur bahaya yang halus seperti \textit{small fault zones} atau \textit{gas pockets} dapat dengan mudah terlewatkan dalam data yang \textit{noisy}, yang menimbulkan risiko signifikan bagi operasi pengeboran. Metode otomatis yang ada sering kali bergantung pada teknik \textit{image processing} tradisional yang kesulitan menangani sifat data seismik yang kompleks dan penuh gangguan.

Makalah ini mengusulkan kerangka kerja deteksi otomatis menggunakan arsitektur \textit{deep learning} YOLO (\textit{You Only Look Once}). Kontribusi utama kami meliputi adaptasi model \textit{object detection} YOLO untuk identifikasi anomali seismik, pengembangan alur \textit{preprocessing} khusus yang melibatkan \textit{percentile-based contrast normalization} dan \textit{overlapping tiling}, serta evaluasi empiris kinerja model dalam mendeteksi berbagai kelas \textit{subsurface hazards}.
